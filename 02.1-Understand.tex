Audit and feedback is first and foremost a way to generate new insights and inform future actions in clinical care. For this reason, the design and implementation of measurement is of critical importance to the creation of effective A\&F. The WHO describes performance measurement as a process which "seeks to monitor evaluate and communicate the extent to which various aspects of the health system meet their key objectives."\cite{smith2009performance} However, not all quality measures are created equal and the creation of suitable and successful metrics is not simple. Current evidence shows that quality measures should be aligned within an organizational context, relevant to its actors, attributable to specific individuals, feasible, accurate, and reliable.\cite{polanczyk2019quality}

For this reason, my first objective is to "develop a set of quality indicators for e-A\&F to assess and improve the appropriateness of heart failure management in adult patients at the MUHC." This set of indicators will then be used in objective 2 and 3 as it will impact (1) data collection, (2) feedback design, and (3) evaluation of the intervention.

\section{Research Design}
I propose an exploratory quantitative analysis of quality measures for heart failure at the MUHC. This study will use two years of retrospective secondary operational data to compare candidate quality measures. Data will be extracted by selecting admitted patients with ICD codes \footnote{From AHA's, Get with the Guidelines\textcopyright - Heart Failure} for heart failure along with the physicians who interacted with them. In order to match the future objective and limit the unnecessary risk to patients, data will be limited to adult patients at the MUHC.

\todo[inline]{Add information here on how many patients/doctors, we approximate this means based on Madjda numbers.}
\todo[inline]{How to justify 2 years (of HF patients, perc. complet in lit) }

Descriptive statistics will be used to provide a tabular overview of the stratified and unstratified measures, their completeness, along with a visual presentation of trends. The stratified exploration will explore the influence of basic patient demographic factors, of their cardiovascular risk, and of physician characteristics. 

\section{Evaluation Criteria}
\begin{itemize}
    \item Campbell's Prerequisites for quality measures (Acceptability, Feasibility, Reliability, Sensitivity, and Predictive Validity) \cite{campbell2002research}
    \item Key considerations when addressing causality and attribution bias (attribution, confounding, and risk adjustment) \cite{terris2009attribution}
    \item Characteristics of effective A\&F (baseline performance, availability of action plan, possible frequency, and latency)\cite{ivers2012audit} % i.e. (asap considering patient load)
\end{itemize}

Baseline performance, possible frequency, latency, attribution, and feasibility will be assessed using the basic statistics mentioned above. Acceptability will be taken into account in objective 2. Sensitivity to change, predictive validity, risk adjustment, and impact of confounders will be assessed using regression modeling. Availability of action plan will be checked manually with their relevant framework.

\section{Candidate Quality Indicators}
Multiple efforts have been made between institutions and advocacy groups to define a set of evidence-based actionable indicators for heart failure.\cite{hong2006overview}\cite{fonarow2010improving} \cite{kelley2006health} At this point, newer intervention are encouraged to use existing list of candidate measures instead of going to the source material. Doing so accelerate development and contributes to the soundness of the evidence, better face/content validity, reproducibility (between institutions), and more evidence on their association with significant health outcomes.\cite{smith2009performance}

The list of candidate retained is the result of a recently compiled systematic review of quality indicators in the ICU.\cite{goldfarb2018systematic} Its set of high-quality indicators will be considered first. But, if too few indicators are meeting the evaluation criteria, more indicator will be chosen from their expended list by local HF experts.

% Quality measures on heart failure can be from four domains: structure (), process (), outcome (), and patient experience (). The choice of quality measures has an important impact on A\&F since, for example, process measures are easier to collect and quicker to influence, but outcome measures are closer to the what healthcare is about. Measures which are stated or derived from more behavioural perspective seems to work better. Finally, quality measures which already have high compliance have, by the ceiling effect, less opportunity for improvement. 