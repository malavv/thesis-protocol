% Data Source and Characteristics.
In this section, I describe the MUHC \gls{DW}, a newly created repository that centralizes data through daily updates from hospital information system used in most care process at the hospital. The \gls{DW} is intended to support both quality improvement and clinical research and currently includes data from systems such as ADT, MedECHO, the emergency department, medical imaging, and the ICU. Summary counts extracted from the MUHC DW for HF patients and treating physicians are presented in table 4.1.

\begin{table}[h!]
\centering
\begin{tabular}{l|rrrrr}
\textbf{Year}       & 2014 & 2015 & 2016 & 2017 & 2018   \\
\textbf{Patients}   & 1637 & 1535 & 1611 & 1739 & 1827 \\
\textbf{Physicians} & 278 & 278 & 266 & 269 & 265 
\end{tabular}
\caption{Counts of HF patients and their physicians, last 5 years, MUHC DW}
\end{table}

In preparing my requests for clinical data sets, I have consulted the detailed data dictionary and met with the DW analyst to ensure that necessary data are available. A request will be made to access records for the cohort of adult patients seen in the emergency department or admitted to the intensive care unit, with patients identified using a previously defined set of ICD codes for HF\footnote{From ``Get with the Guidelines - Heart Failure'', from the American Heart Association}.

I will request access to two datasets. An initial request will be made for access to an anonymized data set for objective one and two. A second data request will be made for identifiable data to conduct the RCT. For the trial, identifiable data are needed as I must ensure the each provider sees the right feedback, and these providers must also be able to see which patient might be in need of additional actions. All data collected by the systems will be linked to anonymous IDs. In case where both a resident and an attending physican are assigned to a patient, both will be considered responsible.

All identifiable data will be maintained on a secure server on the MUHC network. Analyses to create feedback will also be performed on this server. Anonymous data will be kept on a secure password protected computer, where the data analysis will also take place. All identifiable data will be destroyed as soon as the trial is over, and a summary description of the anonymous data will be published along with research articles.

% 1- What does the data look like?
%  - No Schema for the moment, but I can git a flow schema or a data dictionary.
%  - Esp. since some free text extraction will be needed.
%  - Mention in a backup slides that we have already experience extracting data from some echo report? + name and ref. of the guy.
% 2- What is the ref. for the ICD codes? (GWTG)
%  - https://www.heart.org/idc/groups/heart-public/@wcm/@hcm/@gwtg/documents/downloadable/ucm_495599.pdf
% 3 - What about bringing pleasure is important?
%  - Slide/a good answer for the difference between playfulness and gamefulness.

% Cut out
% The McGill University Health Centre Data Warehouse (MUHC DW) began in 2012 with an infrastructure   grant from the Canada Foundation for Innovation (CFI). Now, over 7.5 billion facts are accessible in 380 million rows, and over 6,000 columns, in 381 database tables. These data are linked together and organized for analyses of populations, with over 5.5 million health care episodes as of 2019, and is continually growing.