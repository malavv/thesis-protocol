(1) Research Design (Mixed Methods), remind aim/objective, sample size <one physician at a time, until saturation (when more participants does not results in additional perspective or saturation)>, representativity/generalization have physicians with a variety of characteristics. Alone, in lab, following exemplar tasks (Task Analysis?) using the "Think aloud". The data analysis will be in part done by a medically trained professional. (Optimally a local expert?)

(2) Aim to assess 1) usability, 2) usefulness, and 3) technical feasibility (deployment and collection).

(3) Describe origin of control and experiment system. Describe framework for gamification. (Very lightly describe framework for soft. design)

(4) Obtaining confounder? (Basic demo, role, education, Technology Acceptance Model v3 Form)

(5) Tasks and Scenarios: Fake results, but made to look like real case. Make sure that they feel real. (Describe use tasks|scenarios?)

(6) Additional Questions (1) possible misunderstanding and adverse effects, (2) pain points (3) based on the scenario seen, who you believe you would have followed up with any actions or change in behaviour.

\textbf{Notes: Addition}
\begin{itemize}
    \item \textbf{usability} Missing definition, using SUS https://www.usability.gov/how-to-and-tools/methods/system-usability-scale.html
    \item \textbf{usefulness} (Guideline 1:1 Provide Useful Content <Provide content that is engaging, relevant, and appropriate to the audience>) (usability.gov)
    \item \textbf{Measurement} Describe what is qual and what is quan. Green template string? (Will there be a kind of thematic analysis of the comment, where would the comment be collected?
    \item \textbf{Predict Use} Ask people in the lab setting to predict frequency of use, base on a few given patient rates, and a few given performance scores?
    \item \textbf{Technical Feasibility} Will this be hosted in-hospital, would they have access outside the hospital. Mobile or Browser based. If inside the hospital, are there computer there able to connect/view the page. If yes, would they have/be given time to view the page and explore it. +(Use the same network and IT resources, maybe connect to the same kind of portal? To assess if it scales.)
    \item \textbf{Linkage} How can this be used to inform objective 3? (Ability to measure the metrics used in objective 3)
    \item \textbf{Design of Task/Scenario} What are the scenarios? and how/by who were they defined.
    \item Assess technical feasibility for both.
    \item Mention use of a participant consent.
\end{itemize}

\textbf{Notes: Rejected}
\begin{itemize}
    \item (No focus group) too difficult?
    \item (Only assess the new one) The control will be assumed usable, and useful. Unless it criticizes the quality measures, and in this case it will be modified for both.
\end{itemize}

\textbf{Notes: Verification's}
\begin{itemize}
    \item Real data will elicit more relevant comment and might be the only way to get participant to dig deeply in what is being presented. However, fake data as the advantage of not being so sensitive. Should we choose individuals that would be participants and show them their own data?
\end{itemize}

!! Any statistics that can inform how the RCT will go? (predict frequency of use) !!