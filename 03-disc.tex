A major challenge in designing this electronic audit and feedback system is, in collaboration with the clinical team, find quality improvements topics which are important for the users, have data and quality standards available, and are appropriate targets for A\&F. This also means having clinical partners which are willing to take time to participate in the governance of this project.

Designing, developing, and deploying an e-A\&F system will be time consuming and require expert knowledge in software development and information technology. However, the principal investigators as years of experience in software engineering specifically in large healthcare systems. Additionally, resources and staff from the McGill Clinical and Health Informatics research group will be available for support.

Any ongoing quality improvement effort which overlaps with this project will be a threat to the estimation of generalizable causal effects since it would confound the effect of the intervention.

Finally, since gamification is a holistic strategy which require thinking from the design process. It is first difficult to find a valid comparator, and second difficult to decompose into essential components. This is why we suggest adapting an existing system to the clinical topic chosen instead of building a synthetic comparator. Furthermore, our initial results on the whole system can then inform other teams to decompose it in a more factorial way to see which components are the most important. 

\subsubsection{Strengths and Limitations}

Unintended effects from biases in assessment of performance should be monitored carefully*\cite{terris2009attribution}

\begin{itemize}
    \item RCT results are comparable, easier to use along the lit, and less impacted by bias, in addition to adjusting for measured and unmeasured confounders.
    \item Lab usability made sure the system works.
    \item Team member inside the environment and known (respected)
    \item Can’t predict the effect of main outcome on patient outcome
    \item Unclear what is a clinically significant effect on adoption
    \item The same development will be used throughout the research (multiple studies for same base cost)
    \item Framework for this project will be made open source and available for future development
    \item Given the nature of the environment this is likely to be underpowered, yet multi-centre is unrealistic.
    \item Designers are evaluators, and difficult to create a valid comparator
    \item Missing richness of the experience due to RCT.
    \item Volunteer  bias likely in high performer, which are less likely to improve.
    \item Contamination likely (culture wise, and between individuals)
    \item Mechanism to combat attrition could interact with intervention.
\end{itemize}

\begin{itemize}
    \item Given the nature of the environment (few attending physicians even in large ward), this is likely to be underpowered, yet multi-centre is unrealistic. (20-10-30)
    \item Designers are evaluators (also a problem given we are creating our comparator)
    \item Missing richness of the experience due to RCT.
    \item Some residents will be part-time
    \item Volunteer bias likely in high performer, which are less likely to improve.
    \item Contamination likely (culture wise, and between individuals)
    \item Mechanism to combat attrition could interact with intervention.
\end{itemize}