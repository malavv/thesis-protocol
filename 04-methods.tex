In this section, I describe: 1) the data, 2) a plan for each objective, 3) limitations and  mitigation strategies, and 4) anticipated contributions. The logical thread of the objectives is to understand how to adapt a known strategy to a new context, to develop and validate a prototype solution, and to evaluate the effect of this prototype in this new context. These objectives will be met through a descriptive study to identify suitable quality indicators, a usability study to ensure the gamified system is usable and useful, and a randomized trial to evaluate the effect of gamification on \gls{EAF}.

In the following sections, I describe the design of \gls{AF} the intervention, software engineering practices, and the design and conduct of the \gls{RCT}. In terms of the necessary expertise for the proposed research, I am formally trained in software engineering and have access to the McGill Clinical and Health Informatics software development team. My supervisor is Director of the MUHC data warehouse (DW) and my thesis committee includes Dr. Noah Ivers, a leading worldwide expert on \gls{AF}, and Dr. Robyn Tamblyn, an expert in conducting \gls{RCT}s of computer-based interventions in clinical care settings.