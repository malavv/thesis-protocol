In the next sections, I will describe 1) what data is available, 2) my plan for each objectives, 3) some potential difficulties my plan to mitigate them, and 4) my anticipated contributions to the literature. The overall structure of the objectives is to first, get a better understanding of how to adapt a strategy to a new specific set of conditions; second, develop and validate prototype solutions; third, evaluate the effect of this strategy in this new context. In this study, this will translate to a descriptive study to identify suitable quality indicators, a usability study to ensure the gamified system is usable and useful, and a randomized evaluation of the effect of gamification on \gls{EAF}.

The following sections will describe issues around : software engineering, design of \gls{AF} interventions, and on the design and execution of RCT. It is important to keep in mind that I am formally trained in software engineering and have access to the McGill clinical informatic development team for help. Also, Dr. Noah Ivers, one of the leading worldwide expert and author of the latest Cochrane accepted to participate in my thesis committee. Finally, my thesis committee will also have an expert with experience in RCTs.