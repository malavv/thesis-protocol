\section{Audit and Feedback}
Audit and Feedback (\gls{af}) is widely described as a “summary of the clinical performance of healthcare provider(s) over a specified period of time”\cite{ivers2012audit}. Although inclusive, the broad nature of this definition gives the deceptive impression that A\&F is simple to implement. When taken together with the fact A\&F can lead to small but important effects on health professional behaviour and patient health\cite{ivers2012audit}, it could be misunderstood as saying that any summary of performance will improve practice. Yet, as the latest Cochrane review concluded, “Audit and Feedback continue to be an unreliable approach to quality improvement until we learn how and when it works best” \cite{ivers2012audit}.

The latest research on this subject has focused on developing a theoretical grounding for A\&F \cite{hysong2017theory}\cite{brown2019clinical}\cite{landis2015computer}, identifying characteristics of successful interventions \cite{ivers2012audit}\cite{colquhoun2013systematic}\cite{tuti2017systematic}, and developing evidence-based strategies for adapting and refining A\&F systems \cite{brehaut2016practice}\cite{mcnamara2016confidential}. Through this research, the field has moved towards finding better ways of designing effective audit and feedback interventions. As one might expect, much of this work overlaps with the theories and mechanisms of behaviour change\cite{colquhoun2017advancing}.

While the evidence is evolving, it is unclear how this evidence is being translated into everyday practice. With the increased availability of routinely collected clinical data for the purpose of quality improvements, A\&F interventions are often created in an ad hoc manner. This situation could represent a significant lost opportunity in terms of potential clinical improvements, scientific advances, and better health outcomes.

\section{Motivation and Gamification}
Motivation is a core concept of goal-directed activities along with goal commitment, goal importance, self-efficacy, task complexity, and the availability of feedback \cite{locke2002building}.  There are two types of motivations; intrinsic (or internal) and extrinsic (or external) motivation. Both influence the satisfaction one gets from accomplishing their goal which then affects their willingness to commit to new challenges and future goal commitment \cite{locke2002building}.

Changing practice is in large part about motivating change and consequently there is active research in both the mechanisms that motivate behaviour change \cite{michie2011behaviour} and the impact of A\&F on intentions \cite{gude2018health}. Prior audit and feedback systems have tried multiple schemes to influence the motivations of health professionals. Some systems let users compare themselves against peers \cite{ehrenfeld2014automated}, some reward good performance financially\cite{campbell2007payperf}, and some report performance publicly \cite{schneider1998use}. Yet, although some offered promising results, more research is needed on finding less costly, more easily accepted, or more effective strategies.

One alternative method used by the army for decades, in war games, and increasingly studied explores new ways of utilizing motivation.  Gamification, or “the process of game-thinking and game mechanics to engage users and solve problems” \cite{zichermann2011gamification} could offer A\&F system designers a new way of promoting engagement and provide users a more appealing experience. One author suggests that “When done well, gamification helps align our interests with the intrinsic motivations of our players, amplified with the mechanics and rewards that make them come in, bring friends, and keep coming back.” \cite{zichermann2011gamification}. It is important to distinguish between gamification which merely uses elements of games in a non-game context from “serious games” which describes creating games for non-entertainment purposes \cite{deterding2011game}. It is also important to note that taken individually certain game elements such as goals, competitiveness, and rules might not be by themselves identified as “gameful”. 

Good game design principles often overlap with existing behavioral theory. For example, the need for goals that grow with the player, or progression, is analagous to the usefulness of proximal goals in goal setting theory \cite{locke2002building}. The importance of exploration and discoverability relates to the need for autonomy described in self-determination theory. Finally, game design that fosters a sense of ownership and consequence implicitly favours the creation of personal goals instead of assigned goals, another element of effective goal setting \cite{locke1990theory}. Gamification has been shown in online programs to make users spend more time, contribute more, and lead to downstream behaviour change \cite{looyestyn2017does}.

\section{MUHC Heart Failure}
For these reasons, I propose a research project that seeks to synchronize with clinicians at the McGill University Health Centre, to create a multidisciplinary team tasked with designing, developing, and evaluating a new electronic audit and feedback platform. The clinical quality improvement content of this platform will need to be informed primarily by local needs and goals. This research project aims to lay both \cite{ivers2012audit} the foundation for an effective and scalable A\&F platform for present and future A\&F needs at the \gls{MUHC}, but also \cite{foy2005we} aims to research new hypotheses related to the influence of core behavioural motivators and behaviour change techniques on A\&F.