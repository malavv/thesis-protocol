% Objective 2 : Assess the usability, usefulness, and technical barriers to a gamified \gls{eaf} intervention for physicians at the MUHC.
%
%
The field of \gls{EAF} is still burgeoning and there is a great need for evidence on how to best design \gls{EAF} and on how to maximize its usability using different design decisions.\cite{brown2016interface} Additionally, there is currently no evidence regarding how usable are design decisions which follow from the design of gamified \gls{EAF}. Finally, it would be ill-conceived to embark in the resource intensive process of a randomized trial without having preliminary data on how both systems works.

This is why, in this second objective, I aim to assess the usability, usefulness, and identify technical barriers to a gamified \gls{EAF} intervention for physicians at the \gls{MUHC}. This will result in basic assurance that both arms are receiving interventions with basic efficacy and absent of obvious defects. It will also ensure both systems are able to collect the necessary data for the RCT along with an idea of the technical difficulties related to the deployment of the system in their clinical environment. Finally, this objective will also be used to collect preliminary information aimed at improving the design of objective 3.

% \textbf{Technical Feasibility} Will this be hosted in-hospital, would they have access outside the hospital. Mobile or Browser based. If inside the hospital, are there computer there able to connect/view the page. If yes, would they have/be given time to view the page and explore it. +(Use the same network and IT resources, maybe connect to the same kind of portal? To assess if it scales.)

\section{Research Design}
I will perform laboratory based usability testing of the gamified intervention using a representative convenience sample of five physicians and adding 2 more physicians until saturation is reached. During this evaluation the participants will have to accomplish the same set of tasks, on mocked but realistic data, testing the core functionalities of the system along with those arising from the behavioural intervention.

The choice how many physicians to sample is influenced by the homogeneity of the user group (all high education physicians), the size of the population, and prior evidence exploring the relation between sample size and the number of usability problems uncovered.\cite{nielsen1993mathematical} Saturation is reached when more participants does not results in additional perspectives. Only the experimental system will be assessed for efficiency reason since the control system was already assessed by its authors.\cite{brown2016interface} I assume its adaptation will not change its core usability.

As per the measures, first, the identification of technical difficulties will not be a result of the participants contribution but the research team experience in setting up the test. Then, usability is measured as "the extent to which a system, product or service can be used by specified users to achieve specified goals with effectiveness, efficiency and satisfaction in a specified context of use".\cite{international1998iso} Finally, usefulness is the ability of a system to "provide content that is engaging, relevant, and appropriate to the audience". Both constructs are often used in user experience design and have strong face and content validity.\cite{united2006research}

\section{Procedure}
Participants will first be required to provide informed consent and then fill in a pre-test questionnaire about their general characteristics (10yr age group, gender, specialty, and number of year of clinical practice). Then participants will be presented with a series of tasks to accomplish while "thinking aloud". Think aloud procedure provides greater insight into the users' mental model by asking them to verbalize their thoughts. They were shown to elicit a larger set of usability issue then traditional sessions. \cite{ericsson1980verbal} A researcher will be present and recording the though process along with noting difficulties and confusion as the user performs the tasks. Task performance will be assessed on a scale of 0 to 4, with 0 meaning the user could only accomplish the task with substantial guidance, and 4 meaning the user solved the tasks without help. The provided tasks will be centred around the four central feature of \gls{EAF} (described in the next section) along with additional tasks for coverage of the gamified material. I will create these new tasks following the intervention's design and they will be discussed with my thesis committee.

Following the tasks scenario, participants will be asked to fill in a TAM2 questionnaire and provide any textual unstructured feedback they might want to share at the same time. The TAM2, or technology acceptance model v2, questionnaire is a validated instrument centred on perceived usefulness and perceived ease of use as major determinants of the attitudes and intentions related with Use Behavior.\cite{venkatesh2000theoretical} The TAM2 will collect data on projected adoption, but in addition, I will ask participants to estimate their frequency of use in a combination of scenario with low, medium, and high baseline performance and with low and high rates of HF patients.
% More info on TAM2 Based on Theory of Reasoned Action Sub. Cri. : Computer Anxiety, Comp. Playfulness, Comp. Self-Efficacy, external control.

\section{A\&F Systems}
Both experimental and control branch will use electronic audit and feedback systems following the four key components of \gls{EAF} system interfaces: (1) Summaries of clinical performance; (2) Patient lists; (3) Patient-level data; and (4) Recommended actions.\cite{brown2015meta}

The control system will be based on the user experience and interfaces describes in the PINGR system\cite{brown2016interface}, but adapted for the measures identified in objective 1 and for the collection needed in objective 3. The creator was contacted and offered to help if I have any question. The experimental system will be built by part of the McGill Clinical and Health Informatic (MCHI) software development team. They have years of experience in system development, in usability testing, and in human centered design.\cite{shaban2017pophr}\cite{lavigne2013hybrid}\cite{buckeridge2016developing}

The design of the experimental system will be largely driven by 1) the "GOAL framework" for gamification in software engineering\cite{garcia2017framework}, and 2) the "Rules of Play" the main textbook on game design\cite{salen2004rules}. There are four major category of possible game elements; game dynamics (overarching design), game mechanics (step by step processes), specific components (tangible elements of dynamics and mechanics, and aesthetics (design producing positive affect).\cite{mckeown2016gamification} In order to be kept aligned with project and user specific concerns when designing the system, recommendations from gamified interventions in similar settings and with similar users will be followed.\cite{mckeown2016gamification} A scrum based software development methodology will be used for its flexibility with local evolving needs, its use of short fast-paced development phases, and its use of an embedded product owner, which in this case will be a local expert.\cite{cohn2010succeeding} This methodology enable the use of an agile model of software development.\cite{beck2001manifesto}

\section{Analysis Plan}
The end result of this analysis will be threefold: a per task usability score, a thematically organised set of issues and feedback, and a compound overall usability and usefulness metric. Special attention will be taken to elicit possible misunderstanding of adverse effect. The data will be explored for the influence of demographic factors and how it relates to specific interface elements.

% => How is mock data generated?
% => Real data will elicit more relevant comment and might be the only way to get participant to dig deeply in what is being presented. However, fake data as the advantage of not being so sensitive. Should we choose individuals that would be participants and show them their own data?
% => Contamination of the usability participants?

% Additional References
% => [Usability Website](https://www.usabilitybok.org/usability-testing)