\section{Audit and Feedback}
Audit and feedback (\gls{AF}) is described as a ``summary of the clinical performance of healthcare provider(s) over a specified period of time'' \cite{ivers2012audit}. There is evidence supporting the potential of  \gls{AF} to improve the behaviours of health professionals  and the health of patients \cite{ivers2012audit}. However, despite this broad definition, implementing effective \gls{AF} is not simple. In the latest Cochrane review, the authors concluded that it is not just any summary of performance that will improve practice \cite{ivers2012audit}.

Recent research has focused on the theoretical grounding for A\&F
\cite{hysong2017theory, brown2019clinical, landis2015computer}, identifying characteristics of successful interventions \cite{ivers2012audit, colquhoun2013systematic, tuti2017systematic}, and developing evidence-based strategies for adapting and refining A\&F systems \cite{brehaut2016practice, mcnamara2016confidential}. Through this research, the field is moving towards better ways of designing effective interventions based on theories and mechanisms of behaviour change \cite{colquhoun2017advancing}.

\section{Motivation and Gamification}
Motivation is a core concept of goal-directed activities along with goal commitment, goal importance, self-efficacy, task complexity, and the availability of feedback. There are two types of motivation: intrinsic (or internal) and extrinsic (or external) motivation. Both influence the satisfaction people get from accomplishing their goals which, in turn affects their willingness to undertake new challenges and set future goals  \cite{locke2002building}.

Because changing practice is largely about motivating improvement, there is active research investigating the mechanisms behind behaviour change \cite{michie2011behaviour}  and the impact of A\&F on the intentions of physicians \cite{gude2018health}. Prior audit and feedback systems have tried multiple schemes to influence the motivations of health professionals. Although some offered promising results, more research is needed to find less costly, more easily accepted, and more effective strategies.

Gamification, or ``the process of game-thinking and game mechanics to engage users and solve problems'' \cite{zichermann2011gamification} is an emerging and potentially effective approach to promoting enhanced user engagement and creating more appealing experiences. One author noted that ``When done well, gamification helps align our interests with the intrinsic motivations of our [users], amplified with the mechanics and rewards that make them come in, bring friends, and keep coming back.'' \cite{zichermann2011gamification}. Some elements of gamification such as goals, progression, and scores can already be found in health interventions. In this context, gamification is not about creating games, but rather about leveraging the elements of games that make them appealing \cite{deterding2011game}.

Good game design aligns with theories of behaviour change. For example, the need for goals that grow with the users, or progression, is analogous to the usefulness of proximal goals in goal setting theory \cite{locke2002building}. The importance of exploration relates to the need for autonomy described in self-determination theory. Finally, game design that fosters a sense of ownership and consequence implicitly favours the creation of personal goals instead of assigned goals, another element of effective goal setting \cite{locke1990theory}. Gamification has been shown in online programs to make users spend more time, contribute more, and leads to downstream behaviour change \cite{looyestyn2017does}.

\section{Management of Heart Failure at the MUHC}
As improvements in care help more patients to survive cardiovascular diseases, more people are living with heart failure (\gls{HF}). In 2019, more than 700,000 Canadians live with \gls{HF}\footnote{From the Canadian Chronic Disease Surveillance System, adjusted for the 2019 population}. Providing high-quality care to this population is a concern for the McGill University Health Center (\gls{MUHC}), a leading center for cardiovascular research in Canada, which treats around 1670 patients with \gls{HF} each year. The recent opening of a hospital-wide data warehouse along with a strong cardiovascular research capacity creates an opportunity for improving \gls{HF} care through \gls{AF}.

% How is this background influencing day to day stuff?
%   - IT is more avaiable, increasing use of ad-hoc solutions.
%   - I would argue more evidence is needed on e-A&F, it's creation, and evaluation of elements
%   - Sign. lost opportunity in terms of pot. clin. impro., sci. advances, and better outcomes.