\section{Audit and Feedback}
Audit and Feedback (\gls{AF}) is widely described as a “summary of the clinical performance of healthcare provider(s) over a specified period of time”\cite{ivers2012audit}. Although inclusive, the broad nature of this definition gives the deceptive impression that A\&F is simple to implement. Taken together with evidence on  A\&F potential effects on health professional behaviour and patient health, it could be misunderstood as saying that any summary of performance will improve practice. Yet, as the latest Cochrane review concluded, it is not the case.\cite{ivers2012audit}.

Recent research has focused on theoretical grounding for A\&F \cite{hysong2017theory}\cite{brown2019clinical}\cite{landis2015computer}, identifying characteristics of successful interventions \cite{ivers2012audit}\cite{colquhoun2013systematic}\cite{tuti2017systematic}, and developing evidence-based strategies for adapting and refining A\&F systems \cite{brehaut2016practice}\cite{mcnamara2016confidential}. Through this research, the field moved towards finding better ways of designing effective interventions and predictably discovered many overlaps with the theories and mechanisms of behaviour change\cite{colquhoun2017advancing}.

% While the evidence is evolving, it is unclear how this evidence is being translated into everyday practice. With the increased availability of routinely collected clinical data for the purpose of quality improvements, A\&F interventions are often created in an ad hoc manner. This situation could represent a significant lost opportunity in terms of potential clinical improvements, scientific advances, and better health outcomes.

\section{Motivation and Gamification}
Motivation is a core concept of goal-directed activities along with goal commitment, goal importance, self-efficacy, task complexity, and the availability of feedback. There are two types of motivations; intrinsic (or internal) and extrinsic (or external) motivation. Both influence the satisfaction one gets from accomplishing their goal which then affects their willingness to commit to new challenges and future goals. \cite{locke2002building}

Changing practice is largely about motivating change and consequently there is active research in both the mechanisms that motivate behaviour change \cite{michie2011behaviour} and the impact of A\&F on physicians' intentions\cite{gude2018health}. Prior audit and feedback systems have tried multiple schemes to influence the motivations of health professionals. Yet, although some offered promising results, more research is needed on finding less costly, more easily accepted, or more effective strategies.

Gamification, or “the process of game-thinking and game mechanics to engage users and solve problems” \cite{zichermann2011gamification} could offer A\&F system designers a new way of promoting engagement and providing users with more appealing experiences. One author suggests that “When done well, gamification helps align our interests with the intrinsic motivations of our [users], amplified with the mechanics and rewards that make them come in, bring friends, and keep coming back.” \cite{zichermann2011gamification}. Some elements of gamification such as goals, progression, and scores can already be found in health interventions. However, gamification is not about creating games, but merely leveraging game elements.\cite{deterding2011game}

Good game design principles align with theories of behaviour change. For example, the need for goals that grow with the users, or progression, is analogous to the usefulness of proximal goals in goal setting theory \cite{locke2002building}. The importance of exploration relates to the need for autonomy described in self-determination theory. Finally, game design that fosters a sense of ownership and consequence implicitly favours the creation of personal goals instead of assigned goals, another element of effective goal setting \cite{locke1990theory}. Gamification has been shown in online programs to make users spend more time, contribute more, and lead to downstream behaviour change \cite{looyestyn2017does}.

\section{MUHC Heart Failure}
As improvement in care makes more patient survive cardiovascular diseases, more are living with heart failure (\gls{HF}). In 2019 more than 700,000 Canadians live with \gls{HF}.\footnote{From the Canadian Chronic Disease Surveillance System, adjusted for the 2019 population} This is a concern for the McGill University Health Center (\gls{MUHC}), a leading center for cardiovascular research in Canada which treats around 600 patients with \gls{HF} per year. The recent opening of a hospital-wide data warehouse along with the center's strong research portfolio in cardiovascular care creates an appealing opportunity.

% End