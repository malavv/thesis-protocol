Audit and feedback (\gls{AF}) is an increasingly used quality improvement technique defined as any summary of clinical performance of health care over a specified period of time aimed at providing information to health professionals \cite{flottorp2010using}. \gls{AF} works by reporting discrepancies in clinical practice prompting providers to assess their satisfaction with current performance and to decide whether adjustments are needed \cite{gould2014application}. Physician dashboards and report cards are examples of \gls{AF}.

Institutions use \gls{AF} to identify and encourage best practices among clinicians. \gls{AF} helps them reduce unintended and inappropriate variations in care driven by clinicians who cannot easily or accurately measure their own performance \cite{davis2006accuracy}. Additionally, \gls{AF} is known to be an expandable and relatively inexpensive strategy to promote the uptake of best practices found in high-performing health systems \cite{baker2015creating}. The added attention \gls{AF} is receiving is encouraging as the improvements it provides, even if small, could result in clinically meaningful and cost-effective changes in care processes and health outcomes \cite{ivers2018using}.

Designing effective \gls{AF} remains difficult as there are major gaps in the evidence about how and when it works best \cite{foy2005we}. The \gls{AF} literature explores myriad aspects of implementation, including how care is measured, how findings are communicated, and the mechanisms through which it changes behaviour. As theories of decision-making suggest, one particularly appealing mechanism is the incentivization of certain practices. Research on incentives in \gls{AF} largely targets extrinsic motivators, including monetary \cite{campbell2007payperf}, social \cite{ehrenfeld2014automated}, or reputational \cite{schneider1998use}.  This research has led to encouraging results, but has also identified limitations. In general, researchers have noted that some extrinsic rewards have consistent detrimental effects on intrinsic motivations and therefore on important motivators such as accomplishment, mastery, and/or self-fulfillment \cite{deci1999meta}.

Behavioural techniques to promote intrinsic motivators exist, but their effect on \gls{AF} are unknown. Effective goal setting theory and self-determination theory suggest that incorporation of concepts from both theories could have a clear and beneficial impact on \gls{AF} \cite{locke2002building}, especially through the use of interactive and low-latency summaries such as those used in electronic \gls{AF}. In particular, the strategy of gamification, the use of game design elements in non-game contexts, has the potential to positively impact two important factors of an audit and feedback system: engagement and enjoyment \cite{hamari2014does}.

% 1- What is the evidence it is "increasingly use and supported"?
%   - Support from the WHO, (flotrop), NIH, as part of LHS, NHS, as part of clinical governance
%   - Support from the INESSS, with their priority recommandation.
%   - Numbers difficult to estimate but more research, and greater opportunity everywhere for dashboard, and also just seeing and hearing about them. Most interval QI so, won't be reported.

% Cut out
% In essence, A\&F assesses providers' performance and compares it with established standards to then provide a summary designed to drive improvement \cite{ivers2012audit}.
% As an example, a review of pay-for-performance showed potential short-term improvement in the process of care but little to no effect on intermediate health outcomes \cite{mendelson2017effects}.
% This interplay between external and internal motivators can be characterised as playing the game for the game's own sake. (Nobody understood the ref. correctly, and it was badly worded anyway since I meant "the benefit of focusing on internal instead of external motivators")