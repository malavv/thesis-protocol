Audit and feedback (\gls{AF}) is an increasingly used quality improvement technique defined as a summary of the clinical performance of healthcare providers over a specified period of time. In essence, A\&F assesses providers' performance and compares it with established standards to then provide a summary designed to drive improvement.\cite{ivers2012audit}

There are multiple reasons why some institutions choose to use A\&F. For example, it is known that unintended and inappropriate variation in care are common, and that clinicians are relatively poor at self-assessment.\cite{davis2006accuracy} Additionally, A\&F is known to be a scalable and relatively inexpensive strategy to promote the uptake of best practices found in high-performing health systems.\cite{baker2015creating}. The added attention A\&F receives is encouraging as its improvements, even if small, could result in clinically meaningful and cost effective changes in processes and outcomes.\cite{ivers2018using}

Yet, designing effective A\&F remains difficult. The latest Cochrane reiterated that "[A\&F] will continue to be an unreliable approach to quality improvement until we learn how and when it works best."\cite{foy2005we} This is why the literature explores myriad aspects of its implementation, how it measures care, how it communicates its findings, and through what mechanisms it operates. As theories of decision making suggest, one particularly appealing lever are the  incentives surrounding certain practices. Research on incentives in A\&F has largely been targeted at the use of extrinsic motivators be it monetary\cite{campbell2007payperf}, social\cite{ehrenfeld2014automated}, or reputational\cite{schneider1998use}. This research led to encouraging results but also presented limitations. As an example, a review of pay-for-performance showed potential short term improvement in the process of care but little to no effect effect on intermediate health outcomes.\cite{mendelson2017effects} Authors pointed out some extrinsic rewards have consistent detrimental effects on intrinsic motivations and therefore on important motivators such as accomplishment, mastery and/or self-fulfillment\cite{deci1999meta}. Often characterised as playing the game for the game's own sake.

Behavioural techniques promoting intrinsic motivators exist but their effect on A\&F are unknown. Prior exploration of effective goal setting theory and self-determination theory suggest it could have a clear and beneficial impact on A\&F. Especially through the use of interactive and low-latency summaries such as those used in electronic A\&F. Gamification was shown to positively impact two important factors of an ongoing feedback system; engagement and enjoyment.\cite{hamari2014does}