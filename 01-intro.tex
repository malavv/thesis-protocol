Audit and feedback (\gls{af}) is an increasingly used quality improvement defined as a summary of the clinical performance of healthcare providers over a specified period of time. In essence, A\&F assesses providers' performance and compares it with established standards to then provide a summary designed to drive improvement.\cite{ivers2012audit}

There are multiple reasons why some institutions choose to use A\&F. For example, it is known that unintended and inappropriate variation in care are common, and that clinicians are relatively poor at self-assessment.\cite{davis2006accuracy} Additionally, A\&F is a scalable and relatively inexpensive strategy to promote the uptake of best practices found in high-performing health systems.\cite{baker2015creating}. Furthermore, the added attention for A\&F is encouraging since even small improvement in clinically important processes and outcomes can be meaningful and cost effective, especially when taking into account A\&F's capacity at being implemented across entire jurisdiction.\cite{ivers2018using}

Yet, designing effective A\&F remains difficult. The latest Cochrane reiterated that "[A\&F] will continue to be an unreliable approach to quality improvement until we learn how and when it work best."\cite{foy2005we} This is why the literature explores myriad aspects of: its implementation, how it measures care, how it communicate its findings, and what its mechanisms are. The use of incentives is one strategy that received some attention. These incentives have largely been targeted at the use of external motivators with example like monetary rewards, social comparison, and coercive action. This research lead to encouraging results but also presented many potential limitations. As an example, when the U.K. used financial rewards with GP, it lead to (results) but lead to (unclear outcome) and according to prior research would be hypothesised to lead to a devaluation of the intrinsic motivation of the professionals. Intrinsic motiviation is the motivations that is internal, it is when you do an action for yourself.

Behavioural techniques promoting intrinsic motivators exists but their effect on A\&F haven't been explored yet. Prior exploration of effective goal setting theory and self-determination theory suggest it could have a clear and beneficial impact on A\&F. Especially through the use of interactive and low-latency summaries such as those used in electronic A\&F. Recent research in health interventions showed that one such behavioural technique increased user engagement and improved the fun of using it.
