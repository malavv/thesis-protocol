The aim of this research project is to study the effect of gamification on electronic audit and feedback (e-A\&F) systems. Principally, this will require us to answer four questions:

\begin{itemize}
    \item What targets and measures are most suitable for an e-A\&F intervention, at the CHUM, in cardiology, at this time.
    \item What is the effect of gamification on the adoption of, engagement with, and retention of users of e-A\&F systems
    \item What is the effect of gamification on 
    How does it affect how effective is the base e-A\&F intervention.
    \item How is gamification perceived by the users (and what is this perception's effect)?
    \item (not marketable) How can e-A\&F be designed to offer a gamified experience consistent with the A\&F goals?
\end{itemize}

The central hypothesis of the proposed research is gamification will be an effective and well accepted way of increasing the adoption and engagement of e-A\&F as compared to a traditionally designed system, in medical doctors and trainees. This hypothesis was formulated based on the existing gamification literature and overlaps between the mechanisms of gamification and behaviour change theories. This study protocol will describe how the proposed research is to be conducted, and to ensure the safety of the study participants along with the integrity of the data collected.