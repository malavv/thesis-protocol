The goal of this research is to study the impact that gamification, a behavioural intervention targeting intrinsic motivators, will have on electronic audit and feedback. My hypothesis, is that gamification will be an effective and generally accepted way of increasing the adoption, retention, and engagement with \gls{eaf} when compared with a traditionally designed systems. This hypothesis is supported by existing literature in the field of gamification and on the overlaps in the described mechanisms between gamification and effective theories of behaviour change. Achieving this goal will require focusing on three objectives which I will later describe in a thesis based manuscript:

\begin{enumerate}
    \item Develop a set of quality indicators for \gls{eaf} to assess and improve the appropriateness of heart failure management in adult patients at the MUHC.
    \item Assess the usability, usefulness, and technical barriers to a gamified \gls{eaf} intervention for physicians at the MUHC.
    \item Estimate how gamification affects adoption, engagement, and effectiveness of an e-A\&F intervention.
\end{enumerate}