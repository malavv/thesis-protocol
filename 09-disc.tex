% Study Limitations and Mitigation
A major challenge in designing this electronic audit and feedback system is, in collaboration with the clinical team, finding quality improvements topics which are important for the users, have data and quality standards available, and are appropriate targets for A\&F. This also means having clinical partners which are willing to take time to participate in the governance of this project. These partners are essential when planning an implementation to prevent rejection, lack of trust, or perceived intrusion. Physicians won't be remunerated for their participation, but a request will be made for them to recieve continuing medical education (CME) credits for their participation.

Studying gamification is complicated by the holistic nature of the changes it requires along with the absence of an agreed upon single set of components. For this reason, all the necessary modifications will be evaluated as-a-whole and we will follow an explicit framework while reporting our design choices and their effect. A RCT of an unvalidated intervention could be inefficient, hence objective 2. Given the required difference in design, creating a valid synthetic comparator would be difficult, hence the choice of an existing system. If we observe an effect of the gamified system, future research could decompose the effect of individual elements of the gamified system. Evidence exists pointing to gamified interventions being more fun and marketing could be used to emphasise this aspects and drive adoption. However, both arms will receive the same amount and kind of implementation efforts. The RCT design will not collect richer narrative data on the perception users have of the intervention, but a qualitative exploration could be performed after the trial. It is difficult to forecast the effect of modifications on adoption and engagement and on health-related patient outcomes. Nonetheless, since evidence shows \gls{AF} to be efficacious, an increase in adoption and sustained use should results in improved effectiveness. For the same reason, it is difficult to identify a threshold of clinical meaningfulness.

Also, ongoing quality improvement efforts could threaten the generalizability of the estimated effect of the behavioural intervention. However, because of randomization it is unlikely to affect internal validity. Support staff will be on hand for both systems, and only severe bugs will be fixed during the trial. Secular trends will be difficult to explore but there should be some variation in participants entry time. Contamination is likely to be present with users viewing or sharing reports from another arm. However, since reports are individualized, and the main metrics are adoption and engagement, the effect of contamination is likely to be minimal.

% 1 - What unintended negative effect, and why no stopping rule than?


% Designers are evaluator
% Volunteer bias likely in high performer, which are less likely to improve.
% Mechanism to combat attrition could interact with intervention.