

Beyond quality improvement priorities, audit and feedback systems make uses of metrics and quality measures. The choice of metrics is known to influence performance of the QI system. 

\subsubsection{Measures}

(1) relevant to clinicians and the department, (2) come from established guidelines (from trusted source/evidence) (3) and have functional targets, 
(4) computed from readily available and routinely collected data (5) (tpb) be supported by the organization, be supported by the group, and feel like you have control over its improvement.

For example, process measures are easier to collect and quicker to influence, but outcome measures are closer to the what is important for the patients. Measures which are stated or derived from more behavioural perspective seems to work better. Finally, quality measures which already have high compliance have, by the ceiling effect, less opportunity for improvement.

\subsubsection{Source}

The guideline used to drive this work comes from the American Heart Association. More specifically, its \textit{\href{https://www.heart.org/-/media/files/professional/quality-improvement/get-with-the-guidelines/get-with-the-guidelines-hf/educational-materials/hf-fact-sheet_updated-011119_v2.pdf?la=en&hash=1E39EB095FD5A513D1C3D19B22B48CEE3C6AF4B9}{Get with the guidelines, Heart Failure}} document.

\subsubsection{Process}
Laying an overview of the patient healthcare process. Present also the providers' experience, and their link to the quality metrics.
Provide descriptions of the completeness of the data necessary for the measures, and if they need to be adapted.
Ensure, when using this data set the measure are *conceptually valid* meaning that they aren't representative of an unrelated local characteristic.
If data is missing find whether is can be imputed in some case.
Also identify whether the action and measures can be properly attributed to a single physician.

\subsubsection{Choice}
Choosing in our measures, the measures which are the most relevant for their inclusion in the A\&F using some of the consideration above, help from local partners, and their impact on relevant criteria of audit and feedback. (Such as replication, baseline, latency, and and availability of corrective actions.

\subsubsection{Design}
Retrospective study of (n) patients data for patient seeking treatment for CV at the GLEN, in the Cardio clinic, from 201X-201X. The measures will come from the HF targets documents, categorized in HF Achievment measures, QUAL measures, ... Descriptive statistics will be computed, and basic measures of association to describe the covariance of different related measures. Use of Patient Flow Diagram (what they are good for) to describe the relation between the measures with the care pathway.

\subsubsection{Data}
Data coming from a trustable internal source to the hospital (the clinical data warehouse). Description of the data, and how it is organized.
