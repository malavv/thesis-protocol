First, as more electronic audit and feedback systems are allowed to connect with hospital information system, it becomes primordial to know which validated quality indicators are best suited for uses in automated systems. To the extent care is standardized between institutions, so can we expect of the data collected on it. Hence, this analysis, although local, should be informative to future \gls{HF} interventions as well. 

Second, as mentioned the field of \gls{AF} still struggles to know how and when \gls{AF} works best. This high variability in effect can be seen from a design point of view to emphasize the lack of data on how are individual interface element viewed and perceived. This is why recent articles have put emphasis on studying design elements. This project will contribute to the evidence on the key elements of \gls{EAF} and at the same time, make proposal and evaluate new components of the user experience. 

Third, this project will provide estimates of the effect of gamification on adoption and engagement, providing the literature with an alternative strategy to what has often been expensive reward schemes offering limited benefits. Moreover, limited evidence exist on the effect of gamification when users are highly skilled and educated professionals. Additionally, this research will provide quantitative estimates of the effect modification of gamification on \gls{EAF} systems. 