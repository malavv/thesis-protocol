First, as more \gls{EAF} systems are connected with hospital information systems, implementers need to know which validated quality indicators are best suited for use in automated systems. Since care is relatively standardized between institutions, we can expect similar clinical data. Hence, this analysis, although local, should be generalizable to future \gls{HF} interventions. 

Second, as mentioned, the field of \gls{AF} still struggles to know how and when \gls{AF} works best. This unreliability is due, in part, to knowledge gaps on how specific design elements are view and perceived. This is why recent articles have emphasized studying design elements. This project will contribute to the evidence about the key elements of \gls{EAF} and at the same time, propose and evaluate new components of the user experience. 

Third, this project will provide estimates of the effect of gamification on adoption and engagement, providing evidence about a novel and innovative strategy to harness intrinsic motivators in contrast to prior research on extrinsic motivators, such as expensive reward schemes which have produced limited benefits. Moreover, limited evidence exist on the effect of gamification when users are highly skilled and educated professionals. Finally, this research will also provide quantitative estimates of how gamification modifies the effects of \gls{EAF} systems. 



this high variability in effect can be seen from a design point of view to emphasize the lack of data on how individual interface elements are viewed and perceived.